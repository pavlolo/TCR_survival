\documentclass[12pt]{article}
\usepackage[margin=0.5in]{geometry}
\usepackage{graphicx}
\usepackage{bbold}
\usepackage{amsmath}
\usepackage{amssymb}
\usepackage{amsfonts}
\usepackage{commath}
\usepackage{bm}
\usepackage{hyperref}
\usepackage[makeroom]{cancel}

\begin{document}

\subsection{Estimating species population frequencies from sampled abundances}

Let species $i$ have a population frequency $\rho_i$ and abundance $s_i$ in our sample of mass $S$. For each species we have

\begin{equation}
P(\rho|s, S) = \frac{P(s|\rho, S)}{P(s|S)}P(\rho|S)
\end{equation}

where $P(s|S)=f_s$ is the fraction of mass occupied by specii of size $s$ and $P(\rho|S) = P(\rho)$ is the unknown distribution of population frequencies. Using Poisson distribution with parameter $\lambda=\rho S$ to model $s$ one gets

\begin{equation}
P(\rho|s, S) = \frac{\lambda^s e^{-\lambda}}{s!}f_s^{-1}P(\rho)
\end{equation}

Assuming there are $n$ species and overall $N$ species in the sample one can model the probability to draw a species

\begin{equation}
p \sim Beta(\alpha, \beta)
\end{equation}

on the other hand, the probability to sample species that has population frequency $\rho$ is

\begin{equation}
p = 1 - e^{\rho S}
\end{equation}

Thus we have

\begin{align}
\rho &= -\frac{1}{S} \log 1-p_s \\
p &\sim Beta(n_s, N - n_s)
\end{align}

where $\lambda_i = \rho_i S$. The fraction of sampled species of size $s$ is

\begin{equation}
p_s = \sum_i p_i P(s_i = s)
\end{equation}

\end{document}
